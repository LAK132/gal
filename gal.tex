\documentclass[14pt]{extarticle}

\usepackage{listings}
\usepackage{xcolor}
\usepackage{amsmath,amsfonts}
\usepackage{hyperref}

\title{GEometric Algebra}

\author{LAK132}

\begin{document}

\color{white}
\pagecolor{black!90}

\setlength{\parindent}{0pt}
\setlength{\parskip}{6pt}

\maketitle

\pagebreak
\tableofcontents

%
% Dot product
%

\pagebreak
\section{Dot product}
\label{sec:dot-product}

The dot product (\(\cdot\)) of 2 vectors (\(a\) and \(b\))
creates a scalar:

\newcommand{\dotpart}[4]{ #1_#2 #3_#4 }

\newcommand{\expdot}[3][]{
  #1 \dotpart{#2}{1}{#3}{1} + \cdots + #1 \dotpart{#2}{n}{#3}{n}}

\( a \cdot b = \expdot{a}{b} \)

Note:

\( a \cdot b = b \cdot a \)

%
% Outer product
%

\pagebreak
\section{Outer product}
\label{sec:outer-product}

The outer product (\(\wedge\)) of 2 vectors (\(a\) and \(b\))
creates a bivector (\(\mathbf{A}\)):

\newcommand{\wedgepart}[4]{(#1_#2 #3_#4 - #1_#4 #3_#2) (e_#2 \wedge e_#4)}

\newcommand{\expwedge}[2]{
  \wedgepart{#1}{1}{#2}{2} + \cdots + \wedgepart{#1}{n}{#2}{1}}

\newcommand{\expwedgepre}[3]{
  #1 \wedgepart{#2}{1}{#3}{2} + \cdots + #1 \wedgepart{#2}{n}{#3}{1}}

\newcommand{\expwedgepost}[3]{
  \wedgepart{#1}{1}{#2}{2} #3 + \cdots + \wedgepart{#1}{n}{#2}{1} #3}

\newcommand{\expwedgeprepost}[4]{
  #1 \wedgepart{#2}{1}{#3}{2} #4 + \cdots + #1 \wedgepart{#2}{n}{#3}{1} #4}

\(
  \mathbf{A}
  = a \wedge b \\
  = \expwedge{a}{b}
  % = (a_1 b_2 - a_2 b_1) (e_1 \wedge e_2)
  % + (a_2 b_3 - a_3 b_2) (e_2 \wedge e_3)
  % + \cdots
  % + (a_n b_1 - a_1 b_n) (e_n \wedge e_1)
\)

Where \(n\) is the dimension of \(a\) and \(b\),
and \(e_{1..n}\) are the basis vectors.

If we let
\( A_{e_i e_j} = (a_{e_i} b_{e_j} - a_{e_j} b_{e_i})(e_i \wedge e_j) \),
then this simplifies to:

\(
  \mathbf{A}
  = a \wedge b \\
  = (a_1 e_1 + \cdots + a_n e_n) \wedge (b_1 e_1 + \cdots + b_n e_n) \\
  = a_1 b \\
  = A_{e_1 e_2} + A_{e_2 e_3} + \cdots + A_{e_n e_1}
\)

Chaining:

\(
  a_1 \wedge a_2 \wedge \cdots \wedge a_r
  = \frac{1}{r!} \sum\limits_{\sigma \in \mathfrak{G}_r}
  \textrm{sgn} (\sigma) a_{\sigma (1)} a_{\sigma (2)} \cdots a_{\sigma (r)} ,
\)

Note:

\( a \wedge b = - (b \wedge a) \)

In 2D:

\(
  a \wedge b
  = \mathbf{A}
  = A_{xy}
  = (a_x b_y - a_y b _x) (x \wedge y)
\)

In 3D:

\(
  a \wedge b
  = \mathbf{A}
  = A_{xy} + A_{yz} + A_{zx} \\
  = (a_x b_y - a_y b_x) (x \wedge y)
  + (a_y b_z - a_z b_y) (y \wedge z)
  + (a_z b_x - a_x b_z) (z \wedge x)
\)

\pagebreak
In 3D we also find that the outer product looks very similar to
the cross product:

\(
  a \wedge b = \\
  (a_x b_y - a_y b_x) (x \wedge y) + \\
  (a_z b_x - a_x b_z) (z \wedge x) + \\
  (a_y b_z - a_z b_y) (y \wedge z)
\)

\(
  a \times b = \\
  (a_x b_y - a_y b_x) z + \\
  (a_z b_x - a_x b_z) y + \\
  (a_y b_z - a_z b_y) x
\)

%
% Geometric product
%

\pagebreak
\section{Geometric product}
\label{sec:geometric-product}

The geometric product of 2 vectors (\(a\) and \(b\))
creates a rotor (\(\mathbf{R}\)):

\(
  ab
  = \mathbf{R}
  = \frac{1}{2} (ab + ba) + \frac{1}{2} (ab - ba)
  = (a \cdot b) + (a \wedge b)
\)

For perpendicular vectors (\(c\) and \(d\)) we find that \( c \cdot d = 0 \),
hence:

\( cd = (c \cdot d) + (c \wedge d) = 0 + (c \wedge d) = c \wedge d \)

Because basis vectors are perpendicular,
we can use this to simplify the outer product equation:

\(
  a \wedge b \\
  = (a_1 b_2 - a_2 b_1) e_1 e_2
  + (a_2 b_3 - a_3 b_2) e_2 e_3
  + \cdots
  + (a_n b_1 - a_1 b_n) e_n e_1 \\
  = (a_1 b_2 - a_2 b_1) e_{12}
  + (a_2 b_3 - a_3 b_2) e_{23}
  + \cdots
  + (a_n b_1 - a_1 b_n) e_{n1}
\)

In 2D:

\(
  ab
  = (a_x b_x + a_y b_y)
  + (a_x b_y - a_y b_x) xy
\)

In 3D:

\(
  ab \\
  = (a_x b_x + a_y b_y + a_z b_z)
  + (a_x b_y - a_y b_x) xy
  + (a_y b_z - a_z b_y) yz
  + (a_z b_x - a_x b_z) zx
\)

%
% Reflection
%

\pagebreak
\section{Reflection}
\label{sec:reflection}

Given a vector \(v\) and unit vector \(a\), \(v_\parallel\) is \(v\)
projected onto \(a\):

\( v_\parallel = a (a \cdot v) \)

\(v_\perp\) is the the vector perpendicular to \(a\) that sums with
\(v_\parallel\) to equal \(v\):

\( v_\perp = v - v_\parallel \)

To reflect \(v\) by the plane perpendicular to \(a\),
we define the function \(R_a(v)\):

\( R_a(v) = -ava \)

\(
  va \\
  = (v \cdot a) + (v \wedge a) \\
  = ((v_\parallel + v_\perp) \cdot a)
  + ((v_\parallel + v_\perp) \wedge a) \\
  = (v_\parallel \cdot a) + (v_\perp \cdot a)
  + (v_\parallel \wedge a) + (v_\perp \wedge a) \\
  = (v_\parallel \cdot a) + (0) + (0) + (v_\perp \wedge a) \\
  = (v_\parallel \cdot a) + (v_\perp \wedge a) \\
  = (v_\parallel \cdot a) + (v_\perp a)
\)

\(
  ava \\
  = a (v a) \\
  = a ((v_\parallel \cdot a) + (v_\perp a)) \\
  = (a (v_\parallel \cdot a)) + (a (v_\perp a)) \\
  = (a (v_\parallel \cdot a)) + (a v_\perp a) \\
  = v_\parallel + (a v_\perp a) \\
  = v_\parallel - (v_\perp a a) \\
  = v_\parallel - v_\perp
\)

\(
  -ava \\
  = -(v_\parallel - v_\perp)\\
  = v_\perp - v_\parallel \\
  = v - 2 a (a \cdot v)
\)

%
% Rotation
%

\pagebreak
\section{Rotataion}
\label{sec:rotation}

% Basis vectors table:
%
% 2D:
%   2   1
%
% 3D:
%   2   1
% 123   3
%
% 4D:
%   2   1 123
% 123   3   2
% 134 234   4
%
% 5D:
%   2   1 123 124
% 123   3   2 234
% 134 234   4   3
% 145 245 345   5
%
% 6D:
%   2   1 123 124 125
% 123   3   2 234 235
% 134 234   4   3 345
% 145 245 345   5   4
% 156 256 356 456   6
%
% 7D:
%   2   1 123 124 125 126
% 123   3   2 234 235 236
% 134 234   4   3 345 346
% 145 245 345   5   4 456
% 156 256 356 456   6   5
%   6 126 136 146 156   1

\subsection{2D}
\label{subsec:rotation-2D}

Rotor \( R = R_a + R_{xy} \mathbf{xy} \)

Vector \( v = v_x \mathbf{x} + v_y \mathbf{y} \)

Trivector \( T = R v = R_a v + R_{xy} \mathbf{xy} v \)

\(
  = R_a (v_x \mathbf{x} + v_y \mathbf{y})
  + R_{xy} \mathbf{xy} (v_x \mathbf{x} + v_y \mathbf{y})
\)

\(
  = R_a v_x \mathbf{x} + R_a v_y \mathbf{y}
  + R_{xy} v_x \mathbf{xyx} + R_{xy} v_y \mathbf{xyy}
\)

\(
  = R_a v_x \mathbf{x} + R_a v_y \mathbf{y}
  - R_{xy} v_x \mathbf{y} + R_{xy} v_y \mathbf{x}
\)

\(
  = (R_a v_x + R_{xy} v_y) \mathbf{x}
  + (R_a v_y - R_{xy} v_x) \mathbf{y}
\)

Rotor \( R' = R_a - R_{xy} \mathbf{xy} \)

Vector \( v' = R v R' = T R' \)

\(
  = ((R_a v_x + R_{xy} v_y) \mathbf{x} + (R_a v_y - R_{xy} v_x) \mathbf{y})
  (R_a - R_{xy} \mathbf{xy})
\)

\(
  = ((R_a v_x + R_{xy} v_y) \mathbf{x}
     + (R_a v_y - R_{xy} v_x) \mathbf{y}) R_a \\
  - ((R_a v_x + R_{xy} v_y) \mathbf{x}
     + (R_a v_y - R_{xy} v_x) \mathbf{y}) R_{xy} \mathbf{xy}
\)

\(
  = ((R_a v_x + R_{xy} v_y) \mathbf{x} R_a
     + (R_a v_y - R_{xy} v_x) \mathbf{y} R_a) \\
  - ((R_a v_x + R_{xy} v_y) \mathbf{x} R_{xy} \mathbf{xy}
     + (R_a v_y - R_{xy} v_x) \mathbf{y} R_{xy} \mathbf{xy})
\)

\(
  = ((R_a R_a v_x + R_a R_{xy} v_y) \mathbf{x}
     + (R_a R_a v_y - R_a R_{xy} v_x) \mathbf{y}) \\
  - ((R_a R_{xy} v_x + R_{xy} R_{xy} v_y) \mathbf{xxy}
     + (R_a R_{xy} v_y - R_{xy} R_{xy} v_x) \mathbf{yxy})
\)

\(
  = ((R_a^2 v_x + R_a R_{xy} v_y) \mathbf{x}
     + (R_a^2 v_y - R_a R_{xy} v_x) \mathbf{y}) \\
  - ((R_a R_{xy} v_x + R_{xy}^2 v_y) \mathbf{y}
     - (R_a R_{xy} v_y - R_{xy}^2 v_x) \mathbf{x})
\)

\(
  = ((R_a^2 v_x + R_a R_{xy} v_y) \mathbf{x}
     + (R_a^2 v_y - R_a R_{xy} v_x) \mathbf{y}) \\
  + ((R_a R_{xy} v_y - R_{xy}^2 v_x) \mathbf{x}
     - (R_a R_{xy} v_x + R_{xy}^2 v_y) \mathbf{y})
\)

\(
  = ((R_a^2 v_x + R_a R_{xy} v_y)
     + (R_a R_{xy} v_y - R_{xy}^2 v_x)) \mathbf{x} \\
  + ((R_a^2 v_y - R_a R_{xy} v_x)
     - (R_a R_{xy} v_x + R_{xy}^2 v_y)) \mathbf{y}
\)

\(
  = (R_a^2 v_x + R_a R_{xy} v_y + R_a R_{xy} v_y - R_{xy}^2 v_x) \mathbf{x} \\
  + (R_a^2 v_y - R_a R_{xy} v_x - R_a R_{xy} v_x - R_{xy}^2 v_y) \mathbf{y}
\)

\(
  = (R_a^2 v_x + 2 R_a R_{xy} v_y - R_{xy}^2 v_x) \mathbf{x}
  + (R_a^2 v_y - 2 R_a R_{xy} v_x - R_{xy}^2 v_y) \mathbf{y}
\)

\(
  = ((R_a^2 - R_{xy}^2) v_x + 2 R_a R_{xy} v_y) \mathbf{x}
  + ((R_a^2 - R_{xy}^2) v_y - 2 R_a R_{xy} v_x) \mathbf{y}
\)

\(
  R = \mathbf{xy}
  = (\mathbf{x} \cdot \mathbf{y}) + (\mathbf{x} \wedge \mathbf{y})
  = 0 + 1 \mathbf{xy}
\)

\(
  = ((0^2 - 1^2) v_x + 0 v_y) \mathbf{x}
  + ((0^2 - 1^2) v_y - 0 v_x) \mathbf{y}
\)

\(
  = - v_x \mathbf{x} + - v_y \mathbf{y}
\)

\end{document}